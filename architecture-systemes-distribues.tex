
%----------------------------------------------------------------------------------------
%	DOCUMENT CONFIGURATION
%----------------------------------------------------------------------------------------

\documentclass[utf8,12pt,draft]{article}
\usepackage[a4paper, total={16cm, 26cm}]{geometry}
\usepackage[T1]{fontenc}
\usepackage[francais]{babel}
\usepackage[utf8]{inputenc}
\usepackage{slantsc}
\usepackage[official]{eurosym}
\usepackage[usenames,dvipsnames]{xcolor}
\usepackage{color}
\usepackage{amsmath}
\usepackage{url}
\usepackage{cite}
%\usepackage[backend=bibtex,style=verbose-trad2]{biblatex}



%\definecolor{arsenic}{rgb}{0.23, 0.27, 0.29}

%\setlength{\parindent}{4em}
%\setlength{\parskip}{1em}

%\renewcommand{\familydefault}{\sfdefault}
%\renewcommand{\familydefault}{\sfdefault}
\addto\captionsfrench{\renewcommand{\refname}{R\'ef\'erences Wikip\'edia}}
\renewcommand\refname{R\'ef\'erences Wikip\'edia}

\begin{document}
\pagenumbering{gobble}

\title{\LARGE\bfseries Intoduction \`a l'architecture des syst\`emes distribu\'es}
\author{Joseph Allemandou}
\date{Octobre 2016}
\maketitle

Ce cours a pour objectif de pr\'esenter de fa\c{c}on pragmatique les notions
n\'ecessaires \`a la compr\'ehension des architectures de syst\`emes
distribu\'es. Les exemples utilis\'es sont de complexit\'e croissante :
\begin{itemize}
\item Parall\'elisation d'un algorithme
\item R\'esilience et distribution de charge d'un site web
\item Stockage et calcul distribu\'e --- {\sc Hadoop}.
\end{itemize}

\section{Jour 1 --- Premi\`ere approche}

\subsection{Calculer plus vite - Distribuer les traitements}
\begin{itemize}
\item Ex\'ecution parall\`ele \cite{wikipara} \cite{wikiconcur} \cite{wikimultitache}
\item Threads \& Processus\cite{wikithread} \cite{wikiproc}
\item Synchronisation \cite{wikisync} \cite{wikideadlock}
\item Appronfondissment \cite{wikiordo} \cite{wikicomproc} \cite{wikiverrou}
                        \cite{wikisema} \cite{wikiexclu} \cite{wikisection}
                        \cite{wikiprodcons} \cite{wikilectred}
\end{itemize}


\subsection{La r\'esilience --- L'autre b\'en\'efice}
\begin{itemize}
\item Architecture distribu\'ee \cite{wikiarchidist} --- L'exemple d'un site web
\item B\'en\'efices apport\'es par la r\'eplication \cite{wikirepli} \cite{wikispof} \cite{wikiloadbal}
\item Mod\`eles de r\'eplication \cite{wikimasterslave} \cite{wikimultimaster}
\item Approfondissement \cite{wikicap} \cite{wikiconsensus} \cite{wikipaxos}
                        \cite{wikisplitbrain}
\end{itemize}


\section{Jour 2 --- {\sc Hadoop} \cite{wikihadoop}}

\subsection{G\'en\'eralit\'es et histoire}
\begin{itemize}
\item Passage \`a l'\'echelle \cite{wikiscala} \cite{wikigrille} \cite{wikicalcdist}
\item L'approche historique \cite{wikisuperordi} \cite{wikiomp}
\item La révolution par le web \cite{wikigrappe} \cite{wikicluster}
\item \`A garder en mémoire \cite{wikiillus}
\end{itemize}

\subsection{Architecture du syst\`eme de stockage {\sc HDFS}}
\begin{itemize}
\item Fonctionnement
\item Briques logicielles n\'ecessaires (namenode, datanode)
\end{itemize}


\subsection{Architecture du syst\`eme de gestion de ressources {\sc YARN}}
\begin{itemize}
\item Fonctionnement
\item Briques logicielles n\'ecessaires (resource manager, application master, worker)
\end{itemize}

\subsection{Paradigme de calcul {\sc MapReduce} \cite{wikimapred}}
\begin{itemize}
\item Fonctionnement
\item Points forts et points faibles
\end{itemize}

\subsection{\'Ecosyst\`eme}
{\sc Hive}\cite{wikihive}, {\sc Pig}\cite{wikipig}, {\sc HBase}\cite{wikihbase},
{\sc Spark}\cite{wikispark}


\section{Jour 3 --- Big Data \& \'Evaluation}

\section{Vous avez dit {\sc Big Data} \cite{wikibigdata} ?}
En vrac ... NoSQL\cite{wikinosql}, entrp\^ots de données \cite{wikientrepot}
\cite{wikidatamart}, travail de la donnée \cite{wikiinfodeci} \cite{wikiexplodonnees}
\cite{wikiai} \cite{wikidatasci}

%\printbibliography
\bibliographystyle{unsrt}
\bibliography{biblio}

\end{document}